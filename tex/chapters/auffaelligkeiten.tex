\documentclass[class=scrbook, crop=false]{standalone}
\usepackage[subpreambles=true]{standalone}
\ifstandalone
    \input{../settings+/settings}
\fi

% ----------------------------------------------------------------------------
%                        Auffälligkeiten im Programmcode
% ----------------------------------------------------------------------------
\begin{document}

\chapter{Auffälligkeiten im Programmcode}
\label{ch::Auffaelligkeiten_im_Programmcode}
    Neben den Metriken, die zur Einstufung der Programme im Allgemeinen dienen, ist dennoch auch im Einzelnen die Frage zu stellen,
    ob es Auffälligkeiten wie Fehler oder "unschönen" Programmcode gibt.
    Im Allgemeinen lässt sich aber sagen, dass ChatGPT an vielen Stellen keinen schlechten Code schreibt.
    Gerade ChatGPT 4.0 erklärt den generierten Programmcode meist ausgiebig in den Antworten und fügt auch viele
    Kommentare im Sourcecode hinzu um die Verständlichkeit zu verbessern.

\section{Programmfehler}
\label{sec:programmfehler}
    Auch wenn tatsächlich der Großteil der Programme den richtigen Output lieferten, war das nicht für alle Programme der Fall.
    Die Programme, welche nicht den richtigen Output lieferten, waren entweder gar nicht ausführbar, hingen in einem Deadlock fest, oder funktionierten an sich,
    lieferten tatsächlich eine andere Ausgabe als gewünscht, wobei die Tests größtenteils vorher bekannt waren und in der Aufgabenstellung standen.

\section{Pattern Matching}
\label{sec:pattern_matching}


\section{Dataclasses}
\label{sec:dataclasses}


\end{document}
