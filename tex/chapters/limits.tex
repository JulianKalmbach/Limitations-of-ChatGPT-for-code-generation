\documentclass[class=scrbook, crop=false]{standalone}
\usepackage[subpreambles=true]{standalone}
\ifstandalone
    \input{../settings+/settings}
\fi

% ----------------------------------------------------------------------------
%                               Methods
% ----------------------------------------------------------------------------
\begin{document}

\chapter{Einschränkungen}
\label{ch:einschraenkungen}
    Diese Arbeit lässt auf jeden Fall Rückschlüsse darauf ziehen, wie gut ChatGPT in der jeweiligen Version im Generieren
    von Programmcode ist.
    Allerdings muss ganz klar darauf hingewiesen werden, dass der Datensatz mit 41 Aufgaben sehr klein ist und somit die Ergebnisse leicht verzerrt sein können.
    Um noch präzisere Aussagen über die Effizienz und andere Metriken treffen zu können, braucht es einen Datensatz, der mindestens
    zehnmal so groß ist.
    Des Weiteren sind ein paar der Probleme, die in den Aufgaben zu lösen waren, klassische Programmierprobleme oder Aufgaben (zum Beispiel das Sierpinskidreieck).
    Bei diesen Aufgaben ist es durchaus möglich, dass ChatGPT bereits mit Lösungen zu Aufgaben dieser Art trainiert wurde
    und somit Aussagen darüber, wie gut ChatGPT im Allgemeinen im Programmieren ist, also auch im Bewältigen von neuen Aufgaben,
    zumindest mal mit Vorsicht getroffen werden müssen.

\end{document}
