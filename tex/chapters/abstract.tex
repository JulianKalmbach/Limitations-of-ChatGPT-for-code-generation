\documentclass[class=scrbook, crop=false]{standalone}
\ifstandalone
    \input{../settings+/settings}
\fi

% ----------------------------------------------------------------------------
%                                 Abstract
% ----------------------------------------------------------------------------
\begin{document}

\chapter*{Abstract}
\label{ch:abstract}
{
    % Override language formatting for English abstract
    % Add paragraph indent as English paragraphs must be indented 
    \setlength{\parindent}{2em}
    % Remove paragraph spacing as required in the English language
    \setlength{\parskip}{0em}

    % Replace with your own content
    Seit es ChatGPT gibt, wird es in fast allen Bereichen von einer breiten Masse an Menschen getestet und an seine Grenzen gebracht, oder eben auch nicht.
    Gefährlich kann es allerdings werden, wenn man die Antworten von ChatGPT ungeprüft nutzt oder einfach annimmt, dass diese stimmen würden.
    Zum Einen ist ChatGPT nicht unbedingt darauf ausgelegt eine "richtige" Antwort zu geben, sondern mehr dafür eine in menschlischer Sprache verständliche
    Antwort zu liefern und zum anderen gibt es auch für viele Anwendungsbereiche schlicht noch keine wissenschaftlichen
    Untersuchungen dazu wie zuverlässig ChatGPT auf dem jeweiligen Anwendungsbereich tatsächlich ist.

    In dieser Arbeit habe ich mich mit den Grenzen von ChatGPT bezüglich dem Generieren von Programmcode beschäftigt,
    in dem ich ChatGPT zuerst einmal kleine Codeblöcke erzeugen lassen habe.
    Daraufhin habe ich ChatGPT Aufgaben aus den Übungsblättern der Vorlesung "Einführung in die Programmierung" der Technischen
    Fakultät der Universität Freiburg lösen lassen. Das Bestehen dieser Übungsblätter ist Pflicht für alle Studierenen der
    Informatik. Die Aufgaben sind am Anfang sehr leicht, steigern sich aber stetig und stellen eine gute Überprüfung der
    Fähigkeiten im Umgang mit den Grundlagen der Programmierung dar.
    Da inzwischen zwei Versionen von ChatGPT der Öffentlichkeit zur Verfügung stehen, habe ich beide Versionen, sowohl
    3.5 als auch 4.0 anhand verschiedener Metriken getestet und die Lösungen mit den Musterlösungen verglichen, welche
    vom Lehrstuhl zur Verfügung gestellt wurden.
}
\end{document}
