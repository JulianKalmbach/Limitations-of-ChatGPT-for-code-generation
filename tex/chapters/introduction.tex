\documentclass[class=scrbook, crop=false]{standalone}
\usepackage[subpreambles=true]{standalone}
\ifstandalone
    \input{../settings+/settings}
\fi

% ----------------------------------------------------------------------------
%                                Introduction
% ----------------------------------------------------------------------------
\begin{document}

\chapter{Einleitung}
\label{ch::einleitung}
    Seit es ChatGPT gibt, wird es in fast allen Bereichen von einer breiten Masse an Menschen getestet und an seine Grenzen gebracht, oder eben auch nicht.
    Gefährlich kann es allerdings werden, wenn man die Antworten von ChatGPT ungeprüft nutzt oder einfach annimmt, dass diese stimmen würden.
    Zum einen ist ChatGPT nicht unbedingt darauf ausgelegt eine "richtige" Antwort zu geben, sondern mehr darauf eine in menschlicher Sprache verständliche
    Antwort zu liefern und zum anderen gibt es auch für viele Anwendungsbereiche schlicht noch keine wissenschaftlichen
    Untersuchungen dazu wie zuverlässig ChatGPT auf dem jeweiligen Anwendungsbereich tatsächlich ist.

\section{Motivation}
\label{sec:motivation}
    Auch beim Programmieren wird ChatGPT mehr und mehr genutzt und es ist davon auszugehen, dass von ChatGPT generierte
    Programme immer weiter einzug auch in tatsächlich eingesetzter Software erhält.
    Dabei kann es zu erheblichen Problemen kommen, wenn die Programme von ChatGPT ungeprüft und ohne wissenschaftliche
    Erkenntnisse über die Zuverlässigkeit und Leistungsfähigkeit eingesetzt werden.
    In dieser Arbeit wird dementsprechend versucht dem entgegenzuwirken und Erkenntnisse über die Qualität von durch
    ChatGPT generierten Code zu erlangen.

    Die Kernfragen lauten daher:
    \begin{itemize}
        \item Gibt es Auffälligkeiten beim von ChatGPT generierten Sourcecode und wenn ja welche?
        \item Wie zuverlässig sind die von ChatGPT erzeugten Programme?
        \item Sind die Programme im Vergleich mit der Musterlösung effizient?
    \end{itemize}

\ifstandalone
    \printglossary
    \printbibliography[heading=bibintoc]
\fi

\end{document}
