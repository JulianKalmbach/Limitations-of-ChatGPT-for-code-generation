\documentclass[class=scrbook, crop=false]{standalone}
\usepackage[subpreambles=true]{standalone}
\ifstandalone
    \input{../settings+/settings}
\fi

% ----------------------------------------------------------------------------
%                               Zusammenfassung
% ----------------------------------------------------------------------------
\begin{document}

\chapter{Zusammenfassung}
\label{ch:zusammenfassung}
    ChatGPT kann immer mehr und ist im Umgang mit Programmieraufgaben durchaus ein brauchbares Werkzeug.
    Allerdings ist ChatGPT zum aktuellen Stand noch nicht uneingeschränkt zum Programmieren nutzbar.
    Gerade, wenn man die kostenfreie Version nutzt, muss man sich darüber im Klaren sein, dass diese bestimmte Anforderungen einfach nicht erfüllen kann.
    Und auch im Allgemeinen zeigt sich, dass man derzeit nach wie vor über eigene Programmierfähigkeiten verfügen muss,
    wenn man ChatGPT zum Programmieren verwenden will.
    Der generierte Code ist nicht zwangsweise fehlerfrei und wenn man nicht gerade eine Musterlösung mit dem korrekten
    Output zur Hand hat, kann es durchaus vorkommen, dass Werte als vermeintlich richtig berechnet angesehen werden,
    obwohl sie das gar nicht sind.
    Das ist insbesondere bei sicherheitskritischen Programmen ein großes Problem, weshalb gerade hier ChatGPT definitv
    noch nicht eingesetzt werden kann.
    Auch wenn man besonders effizienten Programmcode erwartet, wird man bei ChatGPT nicht fündig werden und es mag sich
    mehr lohnen, sich an einen erfahrenen Entwickler zu wenden.
    Vergleicht man aber ChatGPT mit Informatik Studierenden im ersten Semester, so ist ChatGPT keineswegs ein schlechtes Werkzeug
    und kann durchaus mit Menschen mithalten.
    Wenn man berücksichtigt, was die Entwicklung von KI der letzten Jahre mit sich gebracht hat, ist da aber vermutlich
    noch viel Luft nach oben und das Potenzial ist noch lange nicht ausgeschöpft.
    Aus diesem Grund ist es auch interessant, Arbeiten wie diese in regelmäßigen Abständen zu wiederholen um, die
    Entwicklung von ChatGPT festhalten zu können.

\end{document}
